\chapter*[Introdução]{Introdução}
\addcontentsline{toc}{chapter}{Introdução}

É de grande importância que profissionais da área da matemática, engenharia e saúde trabalhem em conjunto para criarem novos métodos de identificação de doenças, assim como novas técnicas de  tratamento das mesmas. A utilização de modelos matemáticos para se detectar diversas doenças vem sendo cada vez mais utilizados em todo o mundo, pois podem contribuir fortemente para os estudos e diagnósticos exercidos pelos profissionais da saúde.
Neste contexto, dentre os diversos modelos matemáticos desenvolvidos para simular os processos e fenômenos biológicos têm-se o modelo apresentado por Harry H. Pennes (1948), que descreve o processo de biotransferência de calor nos tecidos vivos. 
De acordo com Pennes (1948), o sistema humano têm a capacidade de  manter a temperatura média interna próxima de $37^{\circ} C$, nesse caso, uma pequena variação dessa temperatura pode indicar a existência de células cancerígenas no local. Portanto, como a Equação de Biotransferência de calor de Pennes descreve o comportamento da temperatura nos tecidos vivos, a sua  solução pode ser utilizada para a  localização de  tumores.

  Para encontrar a solução de equações diferenciais, como a equação de Pennes, o Método da Decomposição de Adomian (MDA) apresenta-se como uma técnica promissora para se resolver de forma mais simples e rápida. Diante do que foi apresentado, o presente trabalho tem como principal objetivo solucionar a Equação de biotransferência de Calor de Pennes via MDA e dar ínicio ao estudo de localizações de tumores de pele. Em primeiro momento, com objetivo de simplificar, o modelo da Equação de Pennes analisado  foi na forma unidimensional e invariável no tempo.
  
  Este trabalho está dividido da seguinte forma:
   
   No capítulo 1 é apresentado um estudo teórico em torno de alguns conceitos importantes sobre o câncer de pele e a transferência de calor em tecidos vivos, também é apresentada a Equação de Biotranferência de calor de Pennes e suas características.
   
   No capítulo 2, o Método da Decomposição de Adomian é demonstrado com detalhes a partir de diversos exemplos.
   
   No capítulo 3, O MDA é aplicado na equação unidimensional e estacionária de Pennes obtendo-se uma solução aproximada da mesma.
   
   No capítulo 4 encontram-se as considerações finais sobre o estudo  realizado e as perspectivas de tarefas a desenvolver na continuidade desse trabalho.
  














