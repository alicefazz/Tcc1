
\chapter{Considerações Finais}


A quantidade de novas pessoas afetadas por cancêr de pele a cada ano foi o que motivou esse estudo. Ainda que seu índice de mortalidade não seja alto em comparação com outros tipos de tumores malignos, a sua ocorrência é extremamente elevada no Brasil, o que faz dessa doença um distúrbio grave de saúde pública.

No presente trabalho aplicou-se o Método da Decomposição de Adomian com objetivo de solucionar de forma aproximada o modelo matemático de Pennes em regime permanente, unidimensional em coordenada cílindricas, tal objetivo foi alcançado satisfatoriamente, mas como pôde ser observado, a resolução do problema sem o uso de programas é bastante extenso e  complexo. 

Diante disso, como perspectiva, é pretendido criar um algoritmo computacional  que refaça tais cálculos. Com isso será possível uma solução númerica mais próxima da realidade, com mais iterações. Também é desejado solucionar a Equação de Pennes via MDA na forma não-linear e tridimensional, tal como é a pele humana. 
